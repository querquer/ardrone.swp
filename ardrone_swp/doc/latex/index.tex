

\subsection*{Einleitung}

Diese Software entstand im Rahmen der Lehrveranstaltung \char`\"{}Softwareprojekt\char`\"{} unter Anleitung der Arbeitsgruppe \char`\"{}Embedded Systems and Operating Systems\char`\"{} an der Otto von Guericke Universität Magdeburg im Sommersemester 2012. Ziel bei diesem Softwareprojekt war es, die AR\_\-Drone von Parrot drei verschiedene Szenarien bewältigen lassen zu können. Dabei diente als Richtlinie, dass die Aufgabe mittels ROS erarbeitet werden sollte. In dieser Dokumentation geht es darum dem Leser die Software und deren Anwendung sowie Komponenten nahe zu bringen. Hierbei werden wir hauptsächlich die Funktionsweise der einzelnen Applikationen erklären, die Probleme und Schwierigkeiten, die dieses Softwareprojekt begleiteten, sowie eine Anleitung zur Inbetriebnahme der Software schildern.

\subsection*{Anforderungen an die Software}

Bei den Szenarien handelt sich um Aufgaben im Bereich \char`\"{}Tracking and Following\char`\"{}, die die Drohne zu absolvieren hat. Dabei haben wir die Software in drei Applikationen unterteilt, jeweils eine Applikation für eine Aufgabe. In der ersten Aufgabe geht es darum, dass die Drohne mit ihrer nach unten gerichteten Kamera eine Linie detektiert und deren Verlauf bis zum Ende folgt. Bei den anderen zwei Herausforderungen geht es darum einen Marker, des Weiteren auch als Tag bezeichnet, zu detektieren und zu folgen. Dabei handelt es sich um zwei Applikationen, da dies zum einen mit ihrer unteren Kamera und zum anderen mit ihrer vorderen Kamera geschehen soll. Die Applikation zur Verfolgung des Tags mit der unteren Kamera soll verwendet werden, um einem Bodenroboter zu folgen.

\subsection*{Softwareumgebung}

Unsere Software baut nicht nur auf ROS, sondern auch auf ein Softwarepaket der Brown Universität auf. In diesem Paket findet man den Treiber ardrone\_\-driver, der die AR\_\-DroneLib an ROS anbindet und so die Drohne erst mittels ROS ansteuerbar macht. Zusätzlich nutzen wir aus diesem Paket das Programm ar\_\-recog, welches zur Marker-\/Erkennung dient. Dem Weiteren wurde uns von der Arbeitsgruppe das Programm TrackLine für die Liniendetektion zur Verfügung gestellt. Den Treiber wie auch TrackLine haben wir an unsere Aufgaben angepasst. Das Programm TrackLine musste an ROS angebunden werden. Bei dem Treiber bestand das Problem, dass nach dem Umschalten von der Frontkamera auf die Bodenkamera am rechten und unteren Bildrand die von der Frontkamera zuletzt gesehenen Teilbilder mitgesendet wurden. Das lag daran, dass die Bodenkamera eine kleinere Auflösung als die Frontkamera hat und für beide der gleiche Puffer benutzt wurde. Da dies zu Problemen bei der Linien-\/ und Tag-\/Erkennung führt, haben wir diesen Fehler behoben. Das Schaubild im Folgenden soll den Zusammenhang unserer Software mit der externen Software veranschaulichen.



\subsection*{Regelung}

Die Kommunikation mit der Drohne geschieht über WLAN. Die Drohne sendet das Kamerabild und die Navigationsdaten, bestehend aus Geschwindigkeit und Rotation, an einen Rechner. Auf dem Rechner läuft der ardrone\_\-driver, der das Kamerabild und die Navigationsdaten als Ros-\/Nachricht verschickt. Bei den Applikationen nutzen wir einen verhaltensgesteuerten Ansatz. Immer wenn eine neue Nachricht von ar\_\-recog bzw. trackline gesendet wird, wird von der jeweiligen Applikation eine Soll-\/Geschwindigkeit ausgegeben. Dabei nutzten die Applikationen zusätzlich die Rotationsdaten der Drohne, um die Position des Tags zu korrigieren, die ar\_\-recog berechnet hat. Dies ist wichtig, da die Drohne beim Ausführen von Bewegungen rotiert. Dadurch kann es beispielsweise passieren, dass ar\_\-recog durch die Rotation ein Tag am oberen Bildrand erkennt und die Drohne aber mittig zum Tag fliegt. Die genaue Umsetzung ist in \hyperlink{namespace_math_a25b9284eb485b732c952786b63343aaa}{Math::pixelDiffBottom(float\& x, float\& y)} beschrieben. Der PD-\/Regler dient der Einbeziehung der aktuellen Geschwindigkeit der Drohne. Unter Verwendung der Soll-\/Geschwindigkeit und der aktuellen Geschwindigkeit gibt der PD-\/Regler eine angepasste Geschwindigkeit aus. Diese wird über den ardrone\_\-driver an die Drohne gesendet. Für jede Applikation wurden unterschiedliche Regelungsparameter verwendet. Diese wurden für die jeweilige Aufgabe angepasst. Bei der Boden-\/Tagverfolgung reagiert die Drohne beispielsweise sehr empfindlich bei der Geschwindigkeitsanpassung in x-\/Richtung, da beim Bodenroboter zu erwarten ist, dass er sich vor und zurück bewegt aber nicht seitlich fährt. Die Regelungsparameter sind in \hyperlink{class_cglobal}{Cglobal} definiert.



\subsection*{Generelle Probleme und Schwierigkeiten}

Die größten Probleme entstanden durch Übertragungsstörungen bei der Kommunikation mit der Drohne über WLAN. Dadurch wurden teilweise bis zu 5 Sekunden keine neuen Navigationsdaten erhalten. Dies ist besonders problematisch für die Regelung. Die Drohne kann sehr schnell ihre Geschwindigkeit ändern, sodass mit 5 Sekunden alten Geschwindigkeitswerten nicht mehr richtig geregelt werden kann. Des Weiteren werden die Rotationsdaten für die Anpassung der Position des Tags verwendet. Wenn dabei veraltete Werte verwendet werden, kommt es ebenfalls zu Fehlern. Ein weiteres Problem, welches durch die Übertragungsstörungen entsteht, ist, dass Steuer-\/Kommandos verspätet und nur teilweise aufgeführt werden. Dabei zeigt die Drohne bei gleichem Programm inkonsistentes Verhalten. Dies hat das Entwickeln der Applikationen erschwert, da bei Tests nicht genau gesagt werden konnte, ob ein unerwünschtes Verhalten der Drohne an einem Fehler im Programm oder durch Übertragungsstörungen entstanden ist. Die Übertragungsstörungen waren raumabhängig. Bei unseren Tests waren sie im Fin-\/Hörsaal am geringsten. Falls starke Übertragungsstörungen auftreten, sind die Applikationen nicht funktionsfähig. Ein weiteres Problem ist, dass die Drohne starken Aufwind erzeugt. Dies hat zur Folge, dass sie in der Nähe von Wänden und besonders im Flur ins Trudeln gerät und somit das Tag bzw. die Linie verliert. Vor allem für die Linienverfolgung ist es problematisch, dass die Drohne nicht gleichzeitig in x-\/Richtung fliegen(nicken) und sich drehen(gieren) kann. Die gewünschte Bewegung wäre der Flug einer Kurve. Allerdings fliegt die Drohne dabei diagonal mit einer hohen Geschwindigkeit nach vorne und kürzt die Kurve damit ab. Dabei fliegt sie sehr schnell, auch wenn in x-\/Richtung nur eine geringe Geschwindigkeit angegeben wurde. Wie das Problem angegangen wurde, wird in der Verhaltensbeschreibung der Linienverfolgung genauer erklärt. Außerdem gab es Bugs in der Firmware der Drohne. Zuerst bestand das Problem, dass die Drohne manchmal ohne ersichtlichen Grund nach oben geflogen ist. Das wurde nach einem Update der Firmware behoben. Danach kam es dazu, dass, falls die Drohne in den Emergency -\/Modus kam, sie die letzten Steuerkommandos nach dem nächsten Start sofort ausgeführt hat. Deshalb sollte man, nachdem die Drohne im Flug in den Emergency -\/Modus kam, einen Reset der Drohne durchführen. Weiterhin ist die Dohne manchmal schon vom Start an sehr unstabil geflogen. Nach einem Neustart ist sie danach meistens wieder normal geflogen. Durch schnelle oder ruckartige Bewegungen der Dohne ruckelt das Kamerabild, sodass die Linie oder das Tag kurzzeitig nicht erkannt werden. In dieser Zeit fliegt die Drohne in die zuletzt gesehene Richtung, sodass dies kein Problem darstellt.

\subsection*{Front-\/Tagverfolgung }

Die Applikation dient zur Tagverfolgung mit der vorderen Kamera. Im Brown-\/Package war dafür bereits eine Implementierung mit dem Namen nolen3d vorhanden. Darin sind keine Höhenanpassung und keine Regelung implementiert. Durch die fehlende Regelung ist die Drohne bei unseren Tests oft abgedriftet und hat das Tag schnell verloren.

\subsubsection*{Verhaltensbeschreibung }

Bei unserer Applikation soll die Drohne einen Abstand von etwa 1,5m zum Tag halten. Im dem Bereich vom 20cm um 1,5m soll sich die Drohne in x-\/Richtung nicht bewegen, damit sie nicht immer in Bewegung ist und besser in y-\/Richtung reagieren kann. Falls das Tag parallel zur Drohne bewegt wird, folgt die Drohne dem Tag parallel. Dafür ist es notwendig, dass sich die Drohne erst anfängt zu drehen, wenn sie einen Winkel von mehr als 24° zum Tag hat. In diesem Fall muss sie sich drehen und ihre y-\/Position anpassen. Dabei ist es problematisch, dass sich die Drohne sehr viel schneller dreht als sie in y-\/Richtung fliegt. Damit sie das Tag bei dieser Bewegung nicht verliert, hört sie auf sich zu drehen, falls das Tag am äußeren Bildrand zu sehen ist. Wenn das Tag durch die Bewegung in y-\/Richtung wieder mittiger im Bild ist, dreht sich die Drohe gegebenenfalls wieder. Weiterhin passt die Drohne ihre Höhe an, falls das Tag am oberen bzw. am unteren Bildrand zu sehen ist. Dabei ist eine Höhenbegrenzung definiert, sodass die Drohne im Bereich von 0,5m bis 1,7m fliegen kann. Falls die Drohne auf das Tag zu fliegt, kippt sie leicht nach vorne und das Tag ist im Kamerabild weiter oben zu sehen. Damit die Drohne dadurch nicht nach oben fliegt, wird mittels der Rotation um die y-\/Achse die Position des Tags korrigiert. Eine genauere Beschreibung ist in \hyperlink{namespace_math_a3fbe7036db847d74ed5f2c21635d02d9}{Math::pixelDiffFront(float\& x, float\& y)} zu finden. Wenn kein Tag zu sehen ist, fliegt die Drohne für 3 Sekunden in die zuletzt gesehene Richtung. Somit folgt die Drohne dem Tag auch in der Zeit, in der kein Tag aufgrund von Ruckeln im Bild wegen schnellen Bewegungen zu sehen ist. Falls das Tag nach 3 Sekunden nicht wiedergefunden wurde, fliegt die Drohne auf der Stelle.

\subsubsection*{Testszenarien }

Wir haben die Applikation im Raum 333 und im Hörsaal der FIN getestet. Bessere Ergebnisse erzielten wir im Hörsaal, da die WLAN-\/Probleme dort geringer waren. Dabei verlor die Drohne das Tag bei der Höhenanpassung und der parallelen Bewegung zum Tag nie. Bei der Bewegung, in der die Drohne sich dreht und gleichzeitig in y-\/Richtung bewegt, kann es in manchen Fällen vorkommen, dass sie das Tag verliert. In seltenen Fällen berechnet ar\_\-recog einen Winkel mit einem falschen Vorzeichen, sodass sich die Drohne in die falsche Richtung dreht.

\subsection*{Linienverfolgung }

Mit dieser Applikation soll die Drohne einer Linie folgen. Zur Erkennung der Linie dient das Programm TrackLine, welches uns von Herrn Dr. Sebastian Zug zur Verfügung gestellt wurde. Dieses haben wir an ROS angebunden, sodass es eine Nachricht immer dann versendet wird, wenn ein neues Bild erhalten wurde. Darin ist die x-\/Position der Linie und der Winkel der Linie zur Drohne angegeben. Die y-\/Position der Linie ist immer in der Bildmitte.

\subsubsection*{Verhaltensbeschreibung }

Wenn die Drohne nach dem Start eine Linie erkennt, steigt sie auf eine Höhe von 0,9 m. Danach beginnt sie der Linie zu folgen. Wie bereits bei den generellen Problemen erwähnt, kann eine Bewegung in x-\/Richtung und gleichzeitige Drehung nicht richtig ausgeführt werden. Deshalb darf die Drohne sich kaum in x-\/Richtung bewegen, wenn sie sich dreht. Oder sie darf sich nur kurz und langsam drehen. Dafür fliegt die Drohne nur in x-\/Richtung, falls sie einen Winkel von weniger als 9° zum Tag besitzt. Weiterhin wird dabei die Geschwindigkeit in y-\/Richtung angepasst, sodass die Linie in der Bildmitte ist. Damit die Drohne keine \char`\"{}Zick-\/Zack\char`\"{} Bewegung ausführt, muss sie sich zur Linie ausrichten. Dabei dreht sie sich sehr langsam, damit der oben genannte Effekt nicht so stark auftritt. Bei einem von Winkel 9° oder größer, versucht die Drohne abzubremsen, bis sie fast keine Geschwindigkeit in x-\/ und y-\/Richtung mehr hat. Danach dreht sie sich bis der Winkel kleiner als 9° ist. Mit dieser Implementierung kann die Drohne keinen Kurven folgen. Stattdessen sollte die Linie Knicke haben. Dabei ist ein Grenzwinkel von 9° gewählt, weil sich die Drohne somit auf geraden Strecken nach vorne bewegt und erst bei Knicken in der Linie stoppt und sich dreht. Wurde ein kleinerer Winkel gewählt, stoppte und richtete sich die Drohne auch teilweise auf geraden Strecken aus, da die Detektion des Winkels nur auf etwa 3° genau ist. Ein größerer Grenzwinkel sollte auch nicht verwendet werden, weil sich sonst die Probleme mit der gleichzeitigen Drehung und Bewegung in x-\/Richtung verschärfen.

\subsubsection*{Testszenarien }

Auch diese Applikation haben wir im Raum 333 und im FIN-\/Hörsaal getestet. Aufgrund der geringen Geschwindigkeiten der Drohne bei der Linienverfolgung, ist ein stabiler Flug der Drohne wichtig. Wie in den generellen Problemen beschrieben, fliegt die Drohne bei manchen Flügen schon ab dem Start nicht stabil. Sollte das der Fall sein muss die Drohne neugestartet werden bevor mit der Linienverfolgung begonnen wird. Bei einem stabilen Flug kann die Drohne der Linie gut folgen und verliert sie nur selten. In der Nähe von Wänden verliert sie dagegen das Tag schneller, da sie dort aufgrund des Windes ins Trudeln gerät.

\subsection*{Bottom-\/Tagverfolgung }

Diese Applikation hat die Aufgabe mit der unteren Kamera ein Tag zu verfolgen, sodass man auf einen mobilen Bodenroboter ein Tag auf der oberen Seite des Körpers befestigen kann und die Drohne dann dem Bodenroboter folgt.

\subsubsection*{Verhaltensbeschreibung }

Nach dem Start der Applikation wird die Drohne als erstes auf ihre übliche Höhe fliegen und von da an versuchen unter sich ein Tag zu detektieren. Sobald sie ein Tag durch ihre untere Kamera entdeckt, wird sie aufsteigen, um einen Abstand von 1,3m zum Tag zu erreichen. Die eigentliche Flugsequenz setzt sich aus einer Rotations-\/ und Translationsbewegung zusammen. Die Rotation wird dafür verwendet, dass die Vorderseite der Drohne immer über der Oberseite des Tags ist. Die Translation hat die Aufgabe den Marker immer im Mittelpunkt des Bildes der unteren Kamera zu fixieren. Sobald die Drohne eine stabile Position über dem Tag besitzt, kann nun das Tag bzw. der Bodenroboter sich bewegen und die Drohne wird daraufhin versuchen ihm zu folgen. Dabei kann es durch die Bewegung der Drohne, wie bei den vorherigen Applikationen, zu Veränderung des erkannten Markers auf dem Kamerabildes führen. Diese verfälsche Position des Markers hab wir korrigiert. Eine genauere Beschreibung ist in \hyperlink{namespace_math_a25b9284eb485b732c952786b63343aaa}{Math::pixelDiffBottom(float\& x, float\& y)} zu finden. Sollte die Drohne das Tag verlieren oder kurzzeitig nicht mehr erkennen, wird sie 3 Sekunden lang in die Richtung fliegen, in der sie es zuletzt sah. Damit versuchen wir gleichzeitig möglichen Bildstörungen vorzukommen. Falls sie das Tag in den 3 Sekunden nicht wiedergefunden hat, wird sie auf der Stelle fliegen.

\subsubsection*{Testszenarien}

Wie die anderen Applikationen haben wir auch diese im Raum 333 als auch im Fin-\/Hörsaal gestestet. Am Anfag der Entwicklung im Raum 333 basierten unsere Tests nur auf einem Marker, den wir mittels Faden bewegen konnten. Als diese Tests erfolgreich absolviert wurden, sind wir auf den Bodenroboter ausgewichen, wobei es keinen signifikanten Unterschied des Verhaltens der Applikation gab. Dabei ist zu sagen, dass die Applikation meistens fehlerfrei nur im Hörsaal funktionierte. Dies lässt sich wohl darauf zurückführen, dass dort die geringste WLAN-\/Störung zu verzeichnet ist. Des Weiteren ist die Positionierung des Markers auf dem Bodenroboter entscheidend. Hier sollte darauf geachtet werden, einen möglichst großen weißen Rand um den Marker zu haben um visuelle Störungen zu unterdrücken. Als zweites sollte der Mittelpunkt des Markers möglichst nah an dem Drehpunkt des Bodenroboters sein. Ansonsten wird die Drohne dazu gezwungen eine starke translatiorische sowie rotatorische Bewegung zu absolvieren, welches häufig zu unkontrollierbaren Flugbewegungen führt.

\subsection*{Weiteres }

Den Installationsguide kann man in \hyperlink{page1}{Installation} finden.

Wie man die Apps startet findet man in \hyperlink{page2}{Start der Apps}. 